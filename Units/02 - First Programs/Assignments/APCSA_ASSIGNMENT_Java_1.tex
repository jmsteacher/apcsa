\documentclass[11pt]{exam}
\usepackage{amsmath,amssymb,enumitem,graphicx,listings,lstautogobble,mathtools,multicol}
\usepackage[usenames,dvipsnames]{color}
\usepackage[paper=a4paper,margin=2.5cm]{geometry}

\title{[AP Computer Science A] Java Assignment \#1}
\newcommand{\CourseName}{AP Computer Science A}
\newcommand{\CourseInitials}{APCSA}
\newcommand{\UnitTitle}{First Programs: An Introduction to Java}
\newcommand{\UnitNumber}{2}
\newcommand{\UnitDoubleNumber}{02}
\newcommand{\AssignmentTitle}{Java Assignment \#1}
\newcommand{\AssignmentNumber}{1}
\newcommand{\AssignmentDoubleNumber}{01}

\lstset{language=Java}
\lstset{showstringspaces=false,showspaces=false,autogobble}

\newcommand{\ColorQuestion}[2]{\renewcommand{\questionlabel}{\colorbox{#1}{\color{white}\bfseries\thequestion}\hfill}\question #2}
\newcommand{\BlueQuestion}[1]{\ColorQuestion{RoyalBlue}{#1}}
\newcommand{\GreenQuestion}[1]{\ColorQuestion{ForestGreen}{#1}}
\newcommand{\YellowQuestion}[1]{\ColorQuestion{Goldenrod}{#1}}
\newcommand{\RedQuestion}[1]{\ColorQuestion{BrickRed}{#1}}
\newcommand{\PurpleQuestion}[1]{\ColorQuestion{RoyalPurple}{#1}}

\setlength{\parindent}{0pt}
\setlist{itemsep=0pt}

\firstpageheader{\CourseName}{}{\CourseInitials.\UnitDoubleNumber.\AssignmentDoubleNumber}
\cfoot{\scshape\footnotesize\CourseName, Unit \#\UnitNumber---\UnitTitle---Assignment \#\AssignmentNumber\\ Woodstock School---Mussoorie, Uttarakhand---India}

\begin{document}
%Assignment Preamble...%
	\begin{center}
		\Large\AssignmentTitle
	\end{center}
%...Assignment Preamble%

%Questions Start...%
	\begin{questions}
		\BlueQuestion{Given the value of \lstinline{a} after the execution of each of the following sequences:}
		\begin{enumerate}
			\item
				\begin{lstlisting}
					int a = 1;
					a = a + a;
					a = a + a;
					a = a + a;
				\end{lstlisting}

			\item
				\begin{lstlisting}
					double a = 2;
					a = a * a;
					a = a * a;
					a = a * a;
				\end{lstlisting}

			\item
				\begin{lstlisting}
					boolean a = true;
					a = !a;
					a = !a;
					a = !a;
				\end{lstlisting}
		\end{enumerate}

		\BlueQuestion{Why does \lstinline{10 / 3} give \lstinline{3} and not \lstinline{3.333333333}? What modifications would you need to make to ensure the value \lstinline{3.333333333}?}
		
		\GreenQuestion{What do each of the following print?}
		\begin{enumerate}
			\item \lstinline{System.out.println(2 + ``bc");}
			\item \lstinline{System.out.println(2 + 3 + ``bc");}
			\item \lstinline{System.out.println((2 + 3) + ``bc");}
			\item \lstinline{System.out.println(``bc" + (2 + 3));}
			\item \lstinline{System.out.println(``bc" + 2 + 3);}
		\end{enumerate}
		
		\GreenQuestion{A physics student gets unexpected results when using the code:}
		\begin{center}
			\lstinline{F = G * mass1 * mass2 / r * r;}
		\end{center}
		to compute values according to the formula $F = Gm_1m_2/r^2$. Explain the problem with the code.

		\YellowQuestion{\emph{Rolling Dice.} Write a program that generates and prints two random integers between $1$ and $6$ (as if you were rolling dice).}
		
		{\small\textbf{Hint:} You can use \lstinline{Math.random()} to generate a random number. Experiment with its output before deciding how you can use it to restrict your values to the desired results.}

		\RedQuestion{\emph{Loan Payments.} Write a program that calculates the monthly payments you would have to make over a given number of years to pay off a loan at a given interest rate, compounded continuously, given the number of years, \lstinline{t}, the principal, \lstinline{P}, and the annual interest rate, \lstinline{r}. The total amount paid at the end of a loan is given by the formula: $A = Pe^{rt}$.}
		
		{\small\textbf{Hint:} Use \lstinline{Math.exp(n)} to calculate $e^n$.}

	\end{questions}
%...Questions End%
\end{document}
